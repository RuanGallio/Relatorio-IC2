% make an article with a4 paper and 12pt font size
\documentclass[a4paper,12pt]{article}
\usepackage{adjustbox}
\usepackage{amsthm}
\usepackage{amsmath} % Import the amsmath package for \text
\usepackage{amssymb} % Import the amssymb package for \mathbb
\usepackage{etoolbox}

\newcounter{definition}[subsection]
\newenvironment{mydef}{%
  \refstepcounter{definition}%
  \textbf{Definição \thesubsection.\thedefinition. }%
  \ignorespaces
}{%
  \par\addvspace{\topsep}
}

\newcounter{theorem}[subsection]
\newenvironment{mytheo}{%
  \refstepcounter{theorem}%
  \textbf{Teorema \thesubsection.\thetheorem. }%
  \ignorespaces
}{%
  \par\addvspace{\topsep}
}

\newcounter{proposition}[subsection]
\newenvironment{prop}{%
  \refstepcounter{proposition}%
  \textbf{Proposição \thesubsection.\theproposition. }%
  \ignorespaces
}{%
  \par\addvspace{\topsep}
}

\newcounter{corollary}[subsection]
\newenvironment{corollary}{%
  \refstepcounter{corollary}%
  \textbf{Corolário \thecorollary. }%
  \ignorespaces
}{%
  \par\addvspace{\topsep}
}


% make a demonstration environment. Add black square at the end of the demonstration
\newenvironment{demonstration}{%
  \par\addvspace{\topsep}
  \textbf{Demonstração. }%
  \ignorespaces
}{%
  \par\addvspace{\topsep}
  \hfill $\blacksquare$
}

% This code ensures that theorem and definition counters are reset with every new section
\makeatletter
\@addtoreset{theorem}{subsection}
\@addtoreset{definition}{subsection}
\@addtoreset{proposition}{subsection}
\@addtoreset{corollary}{theorem}
\makeatother


\begin{document}

% align right
\begin{flushleft}
    \textbf{\large PRÓ-REITORIA DE PESQUISA E PÓS-GRADUAÇÃO}
    \linebreak
    \linebreak
    \textbf{\large COORDENAÇÃO LOCAL DO PROGRAMA DE INICIAÇÃO CIENTÍFICA}
\end{flushleft}

\begin{center}
    \textbf{RELATÓRIO}

    \textbf{ANEXO I}
\end{center}

\section{Resumo}

Neste trabalho pretendemos apresentar um resultado de existência e unicidade de
solução para um problema envolvendo a equação da onda. Será estudado classes de
problemas com dados iniciais regulares. Precisamente, consideraremos os dados
iniciais em espaços de funções m vezes continuamente diferenciáveis. Para isso será
efetuado levantamento bibliográfico preliminar sobre tópicos de Análise Real e
Espaços Métricos. Serão estudados aspectos relativos ao espaço das funções
contínuas com métricas apropriadas. O resultado principal será provado via Séries de
Fourier.

\section{Objetivos}

\begin{itemize}
    \item Complementar a formação do discente, visando um perfil científico.

    \item Estudar aspectos relativos aos espaços das funções contínuas, e das funções
          continuamente diferenciáveis, sob o ponto de vista de espaços métricos.

    \item Provar um teorema que estabeleça condições para a existência de solução
          para a equação de propagação de ondas acústicas.

    \item Propiciar que o aluno aprenda aspectos teóricos sobre Análise Real, Espaços
          Métricos e aplicá-los no estudo da equação da onda.

    \item Apresentar os resultados finais no Encontro Anual de Iniciação Científica,
          Tecnológica e Inovação.
\end{itemize}

\section{Cronograma de atividades}

% Make a table from september 2023 to August 2024 as colluwns and 5 rows.
% Add each month as title of the column and the activities as rows.

\begin{adjustbox}{width={\textwidth},totalheight={\textheight},keepaspectratio}%
    \centering
    \begin{tabular}{|c|c|c|c|c|c|c|c|c|c|c|c|c|}
        \hline
        \textbf{Atividades}          & \textbf{Set} & \textbf{Out} & \textbf{Nov} & \textbf{Dez} & \textbf{Jan} & \textbf{Fev} & \textbf{Mar} & \textbf{Abr} & \textbf{Mai} & \textbf{Jun} & \textbf{Jul} & \textbf{Ago} \\
        \hline
        Estuda das referências       & X            & X            & X            & X            & X            & X            & X            & X            &              &              &              &              \\
        \hline
        Escruta do relatório parcial &              &              &              &              & X            & X            &              &              &              &              &              &              \\
        \hline
        Formulação do resultado      &              &              &              &              &              &              & X            & X            & X            & X            &              &              \\
        \hline
        Prova do resultado           &              &              &              &              &              &              &              & X            & X            & X            & X            &              \\
        \hline
        Escrita do relatório final   &              &              &              &              &              &              &              &              &              & X            & X            & X            \\
        \hline
    \end{tabular}
\end{adjustbox}


\section{Materiais e métodos}
Como trata-se de pesquisa teórica, inicialmente o aluno deverá realizar um
levantamento bibliográfico de modo a ter subsídios teóricos suficientes para
demonstrar o resultado principal. Basicamente deve envolver aspectos relativos à
Análise Real e Espaços Métricos. Para isto, o discente fará estudos individuais das
referências bibliográficas e deverá apresentar seminários semanais ao orientador.
Estes seminários serão realizados em sala de aula com a exposição em quadro ou
remotamente. Além dos encontros nos seminários o aluno deverá se reunir com o
orientador, sempre que necessário, para discussão e análise dos resultados obtidos.


Os estudos preliminares compreenderão uma revisão contendo resultados de
Análise Real e tópicos preliminares de Espaços Métricos. Isto será feito do um ponto
de vista teórico e seguirá o contido nas referências White (1993), Lima (1976), Lima
(1977), Lima (2002), Medeiros et al. (2011), Rudin (1975) e Kreyszig (1978). Um dos
principais objetivos é estudar aspectos relativos ao espaço das funções contínuas. Tal
espaço será estudado do ponto de vista de espaços métricos e será explorado suas
características com diferentes métricas. O segundo principal objetivo é explorar as
características dos espaços das funções contínuas, e das funções m vezes
continuamente diferenciáveis, e provar um teorema que garanta a existência de
solução para um problema de valores iniciais envolvendo a equação da onda.

\section{Descrição dos principais resultados}

\subsection{Teorema de existência de solução da equação da onda}
Nesta subseção, apresentaremos definições bases sobre Equações Diferenciais Parciais (EDP), em específico, a equação da onda e o fundamento teórico
necessário para a prova do teorema de existência e unicidade de solução para o problema de valor inicial e de fronteira (PVIF) associado à equação da onda.


\begin{mydef}
    Uma função $f: \mathbb{R} \to \mathbb{R}$ é \textit{periódica} de período $T$ se, e somente se, $f(x + T) = f(x)$ para todo $x \in \mathbb{R}$.
\end{mydef}

\begin{mydef}
    Dada uma série de funções $\sum_{n=1}^{\infty} u_{n}(x)$ onde $u_{n}(x): I \to \mathbb{R}$ \textit{converge pontualmente} se, para cada $x_0 \in I$ fixado,
    a série numérica $\sum_{n=1}^{\infty} u_{n}(x_0)$ é convergente. Ou seja, dados $\epsilon > 0$ e $x_0 \in I$, existe $N(\epsilon, x_0) \in \mathbb{N}$ tal que
    \begin{equation}
        \left| \sum_{j=n}^{m} u_{j}(x_0) \right| < \epsilon
    \end{equation}
    para todo $m \geq n \geq N(\epsilon, x_0)$.
\end{mydef}

\begin{mydef}
    Dada uma série de funções $\sum_{n=1}^{\infty} u_{n}(x)$ onde $u_{n}(x): I \to \mathbb{R}$ \textit{converge uniformemente} se, para cada $\epsilon > 0$, existe
    $N(\epsilon) \in \mathbb{N}$ (ou seja, independente de $x \in I$) tal que
    \begin{equation}
        \left| \sum_{j=n}^{m} u_{j}(x) \right| < \epsilon
    \end{equation}
    para todo $m \geq n \geq N(\epsilon)$ e para todo $x \in I$.
\end{mydef}

\begin{mytheo}
    \textbf{Teste M de Weierstrass:} Seja $\sum_{n=1}^{\infty}u_{n}(x)$ uma série de funções $u_{n}: I \to \mathbb{R}$ definidas em um subconjunto $I$ de $\mathbb{R}$. Suponha que existam constantes $M_{n}$ tais que
    \begin{equation}
        |u_{n}(x)| \leq M_{n}|, \qquad \text{para todo  }x \in I,
    \end{equation}

    e que a série numérica $\sum_{n=1}^{\infty}M_{n}$ convirja. Então a série de funções  $\sum_{n=1}^{\infty}u_{n}(x)$ tem convergência uniforme em $I$.
\end{mytheo}

\begin{mydef}
    Dada uma função $f: \mathbb{R} \to \mathbb{R}$, periódica, de período $2L$, integrável e absolutamente integrável, podemos expressá-la como uma \textit{série de Fourier}
    da seguinte forma
    \begin{equation}
        f(x) = \frac{1}{2}a_0 + \sum_{n=1}^{\infty} a_{n}\cos\left(\frac{n\pi x}{L}\right) + b_{n}\sin\left(\frac{n\pi x}{L}\right).
    \end{equation}
\end{mydef}

\begin{mydef}
    Dada uma função $f$ que possa se expressa como um série de Fourier, os coeficientes $a_{n}$ e $b_{n}$ são chamados de \textit{coeficientes de Fourier} de $f$ e são dados por
    \begin{align}
        a_{n} & = \frac{1}{L}\int_{-L}^{L}f(x)\cos\left(\frac{n\pi x}{L}\right)\, dx; \quad n \geq 0 \\
        b_{n} & = \frac{1}{L}\int_{-L}^{L}f(x)\sin\left(\frac{n\pi x}{L}\right)\, dx; \quad n \geq 1
    \end{align}
\end{mydef}


\begin{mydef}
    A equação da onda é dada pelo Problema de Valor Inicial e de Fronteira (PVIF)

    \begin{equation}
        \begin{cases}
            u_{tt} = c^{2}u_{xx}, & \text{em } \mathbb{R}      \\
            u(0,t) = u(L,t) = 0,  &                            \\
            u(x, 0) = f(x),       & \text{em } 0 \leq x \leq L \\
            u_t(x, 0) = g(x),     & \text{em } 0 \leq x \leq L
        \end{cases}
    \end{equation}
\end{mydef}

\begin{mydef}
    Uma função $f: \mathbb{R} \to \mathbb{R}$ é \textit{seccionalmente diferenciável} se ela for seccionalmente contínua e se a sua derivada $f'$ for seccionalmente contínua.
\end{mydef}

\begin{mytheo}
    \textbf{Teorema de Fourier:} Seja $f: \mathbb{R} \to \mathbb{R}$ uma função seccionalmente diferenciável de período $2L$. Então, a série de Fourier de $f$ converge em cada ponto $x$ para
    \begin{equation}
        \lim_{h \to 0} \frac{1}{2}[f(x+h) + f(x-h)]
    \end{equation}
\end{mytheo}

\begin{mydef}
    Uma função $f: \mathbb{R} \to \mathbb{R}$ é \textit{par} se $f(x) = f(-x)$ para todo $x \in \mathbb{R}$.
\end{mydef}

\begin{mydef}
    Uma função $f: \mathbb{R} \to \mathbb{R}$ é \textit{ímpar} se $f(x) = -f(-x)$ para todo $x \in \mathbb{R}$.
\end{mydef}

\begin{prop}
    Seja $f: \mathbb{R} \to \mathbb{R}$ uma função par integrável em qualquer intervalo $[-L, L]$. Então

    \begin{equation}
        \int_{-L}^{L} f(x) \, dx = 2\int_{0}^{L} f(x) \, dx.
    \end{equation}

    Portanto, se $f$ é par e de período $2L$, então os coeficientes de Fourier são dados por
    \begin{align*}
        a_n & = \frac{2}{L}\int_{0}^{L}f(x)\cos\left(\frac{n\pi x}{L}\right)\, dx, \\
        b_n & = 0
    \end{align*}

\end{prop}

\begin{prop}
    Seja $f: \mathbb{R} \to \mathbb{R}$ uma função ímpar integrável em qualquer intervalo $[-L, L]$. Então

    \begin{equation}
        \int_{-L}^{L} f(x) \, dx = 0
    \end{equation}

    Portanto, se $f$ é ímpar e de período $2L$, então os coeficientes de Fourier são dados por
    \begin{align*}
        a_n & = 0,                                                                \\
        b_n & = \frac{2}{L}\int_{0}^{L}f(x)\sin\left(\frac{n\pi x}{L}\right)\, dx
    \end{align*}

\end{prop}

\begin{prop}
    \textbf{Desigualdade de Bessel:} Seja $f: \mathbb{R} \to \mathbb{R}$ uma função quadrado integrável em $[-L, L]$ (ou seja, $\int_{-L}^{L} | f(x) | \, dx $ exista).
    e $a_n, b_n$ sejam seus coeficientes de Fourier. Então
    \begin{equation}
        \frac{1}{2}a^{2}_{n} + \sum_{n=1}^{\infty}(a^{2}_{n} + b^{2}_{n}) \leq \frac{1}{L}\int_{-L}^{L} | f(x) |^{2} \, dx
    \end{equation}
\end{prop}

\begin{mytheo}
    \textbf{Teorema de existência de solução da equação da onda:} Dado o problema de valor inicial e de contorno
    \begin{equation*}
        \begin{cases}
            u_{tt} = c^{2}u_{xx}, & \text{em } \mathbb{R}      \\
            u(0,t) = u(L,t) = 0,  & t \geq 0                   \\
            u(x, 0) = f(x),       & \text{em } 0 \leq x \leq L \\
            u_t(x, 0) = g(x),     & \text{em } 0 \leq x \leq L
        \end{cases}
    \end{equation*}

    Sendo $f$ e $g$ funções contínuas em $[0,L]$ tais que $f, f', f''$ e $g, g'$ são contínuas e
    $f'''$ e $g''$ sejam seccionalmente contínuas. Seja $f(0)=f(L)=f''(0)=g(0)=g(L)=0$ então

    \begin{enumerate}
        \item $a_n, b_n$ estão bem definidos em
              \begin{align}
                  a_n & = \frac{2}{L}\int_{0}^{L}f(x)\cos\left(\frac{n\pi x}{L}\right)\, dx,     \\
                  b_n & = \frac{2}{n\pi c}\int_{0}^{L}g(x)\sin\left(\frac{n\pi x}{L}\right)\, dx
              \end{align}
        \item Ocorre que
              \begin{align}
                  f(x) & = \sum_{n=1}^{\infty} a_n \sin\left(\frac{n\pi x}{L}\right)                   \\
                  g(x) & = \sum_{n=1}^{\infty} \frac{n \pi c}{L} b_n \sin\left(\frac{n\pi x}{L}\right)
              \end{align}
        \item A solução do problema é dada por
              \begin{equation}
                  u(x,t) = \sum_{n=1}^{\infty} \left[ a_n \cos\left(\frac{n\pi ct}{L}\right) + b_n \sin\left(\frac{n\pi ct}{L}\right) \right] \sin\left(\frac{n\pi x}{L}\right)
              \end{equation}
              onde $u(x,t)$ é uma função contínua em $\mathbb{R}$, de classe $C^{2}$.
    \end{enumerate}
\end{mytheo}


\subsection{Cardinalidade de conjuntos}


Nesta subseção, definiremos os conceitos de conjuntos finitos e infinitos, e apresentaremos os conceitos de cardinalidade de conjuntos e, com isso, a distinção
entre conjuntos numeráveis e não numeráveis. Para tal, definiremos e tomaremos como partida o conjunto dos números naturais $\mathbb{N}$.

\vskip 0.5cm

\begin{mydef}
    Dada um conjunto $\mathbb{N}$ e uma função $s:\mathbb{N} \to \mathbb{N}$, chamamos os elementos
    desse conjunto de números naturais se, e somente se, $s$ satisfaz os seguintes axiomas \textit{(chamados Axiomas de Peano)}
    \begin{enumerate}
        \item $s: \mathbb{N} \to \mathbb{N}$ é uma função injetora.
        \item $\mathbb{N}\backslash s(\mathbb{N})$ consta de um só elemento; tal elemento é chamado de "um", com símbolo $1$.
              Ou seja, 1 não é sucessor de nenhum número natural.
        \item Se $X \subset \mathbb{N}$ é um subconjunto tal que $1 \in \mathbb{N}$ e,
              para todo $n \in \mathbb{N}$, se $n \in X$, então $s(n) \in X$, então $X = \mathbb{N}$.
    \end{enumerate}
\end{mydef}

\begin{mydef}
    Em $\mathbb{N}$, definimos a adição de dois números naturais como a operação $+$ sobre um par $(m,n)$ da seguinte forma:
    \begin{enumerate}
        \item $m + 1 = s(m)$
        \item $m + s(n) = s(m + n)$, ou seja, $m + (n+1) = (m+n)+1$
    \end{enumerate}
\end{mydef}


\begin{mydef}
    Em $\mathbb{N}$, definimos a multiplicação de dois números naturais como a operação $\cdot$ sobre um par $(m,n)$ da seguinte forma:
    \begin{enumerate}
        \item $m \cdot 1 = m$
        \item $m \cdot (n+1) = m\cdot n + m$
    \end{enumerate}
\end{mydef}

\begin{mytheo}
    Sobre as operações de adição e multiplicação de números naturais, temos as seguintes propriedades:
    \begin{itemize}
        \item \textit{associatividade}: $(m + n) + p = m + (n + p)$ e $(m \cdot n) \cdot p = m \cdot (n \cdot p)$
        \item \textit{comutatividade}: $m + n = n + m$ e $m \cdot n = n \cdot m$
        \item \textit{distributividade}: $m \cdot (n + p) = m \cdot n + m \cdot p$
        \item \textit{lei do corte}: $m + p = n + p \Rightarrow m = n$ e $m \cdot p = n \cdot p \Rightarrow m = n$
    \end{itemize}
\end{mytheo}

\begin{mytheo}
    \textbf{Princípio da boa-ordenação} Todo subconjunto não vazio $A \subset \mathbb{N}$ possui um menor elemento.
\end{mytheo}

\begin{mydef}
    Definimos como o conjunto de inteiros menores ou iguais a $n$ como $I_n = \{ p \in \mathbb{N}; p \leq n \}$.
\end{mydef}

\begin{mydef}
    Um conjunto $X$ é dito finito se é vazio ou, então, se existe uma função bijetora $f: I_n \to X$.
    A bijeção $f$ chama-se \textit{contagem} de $X$ e o número $n$ chama-se \textit{número de elementos} ou \textit{número cardinal} de $X$.
\end{mydef}

\begin{mytheo}
    Seja $X$ um conjunto finito e $f:X \to X$ uma função. Então, $f$ é injetora se, e somente se, $f$ é sobrejetora.
\end{mytheo}

\begin{mydef}
    Um subconjunto $X \subset \mathbb{N}$ é dito \textit{limitado} quando existe $p \in \mathbb{N}$ tal que $x \leq p, \forall x \in X$.

    \begin{corollary}
        Um subconjunto $X$ de $\mathbb{N}$ é finito se, e somente se, $X$ é limitado.
    \end{corollary}
\end{mydef}


\begin{mydef}
    Um conjunto $X$ é dito infinito se não é finito.
    Ou seja, $X$ é infinito se, e somente se, não existe uma função bijetora $f: I_n \to X$ para nenhum $n \in \mathbb{N}$.
\end{mydef}

\begin{mytheo}
    Se $X$ é um conjunto infinito, então existe uma função injetora $f: \mathbb{N} \to X$.

    \begin{corollary}
        Um conjunto $X$ é infinito se, e somente se, existe uma bijeção $\psi: X \to Y$ sobre um subconjunto próprio $Y \subset X$.
    \end{corollary}
\end{mytheo}

\begin{mydef}
    Um conjunto $X$ é dito \textit{enumerável} quando é finito ou quando existe uma bijeção $f: \mathbb{N} \to X$.
    Neste caso, $f$ chama-se \textit{enumeração} de $X$.
\end{mydef}

\begin{mytheo}
    Todo subconjunto $X \subset \mathbb{N}$ é enumerável.

    \begin{corollary}
        O produto cartesiano de dois conjuntos enumeráveis é enumerável.
    \end{corollary}

\end{mytheo}


\subsection{Números reais}

Nesta subseção, exibiremos as definições de corpo, corpo ordenado e corpo ordenado completo, e, em seguida,
exibiremos um axioma que garante a existência de um corpo ordenado completo, chamado de corpo dos números reais, e apresentaremos algumas de suas propriedades
e teoremas relacionados com tal corpo.

\vskip 0.5cm

\begin{mydef}
    Uma terna $(K, +, \cdot)$ onde $K$ é um conjunto e $+, \cdot$ são operações é um \textit{corpo} se, e somente se, satisfaz as seguintes propriedades:

    \begin{enumerate}
        \item \textit{Associatividade:} para todo $a, b, c \in K$, temos que $(a + b) + c = a + (b + c)$ e $(a \cdot b) \cdot c = a \cdot (b \cdot c)$.
        \item \textit{Comutatividade:} para todo $a, b \in K$, temos que $a + b = b + a$ e $a \cdot b = b \cdot a$.
        \item \textit{Elemento neutro:} existem $0, 1 \in K$ tais que $a + 0 = a$ e $a \cdot 1 = a$ para todo $a \in K$.
        \item \textit{Distributividade:} para todo $a, b, c \in K$, temos que $a \cdot (b + c) = a \cdot b + a \cdot c$.
        \item Todo $a \in K$ possui um \textit{oposto} $-a$ tal que $a + (-a) = 0$.
        \item Todo $a \in K \backslash \{0\}$ possui um \textit{inverso} $a^{-1}$ tal que $a \cdot a^{-1} = 1$.
    \end{enumerate}
\end{mydef}

\begin{mydef}
    Um corpo $(K, +, \cdot)$ é dito \textit{ordenado} se existe um subconjunto $P \subset K$, chamado de conjunto dos elementos positivos, tal que
    \begin{enumerate}
        \item A soma de dois elementos positivos é um elemento positivo: $a, b \in P \Rightarrow a + b \in P$ e $a, b \in P \Rightarrow a \cdot b \in P$.
        \item Dado $a \in K$ existem três possibilidades: $a \in P$, ou $-a \in P$ ou $a = 0$.
    \end{enumerate}
\end{mydef}

\begin{mydef}
    Em um corpo ordenado $(K, +, \cdot)$, podemos definir uma relação de ordem $<$ e dizemos que \textit{x é menor que y} se $x<y \iff y - x \in P$.
\end{mydef}


\begin{mydef}
    Dado um corpo ordenado $(K, +, \cdot)$, definimos como \textit{valor absoluto (ou módulo)} de um elemento $a \in K$ como $|0| =0$ e $|a| = a$ se
    $a \in P$ e $|a| = -a$ se $-a \in P$.
\end{mydef}

\begin{mydef}
    Dado um corpo ordenado $(K, +, \cdot)$, um subconjunto $A \subset K$ é dito \textit{limitado superiormente} se existe $M \in K$ tal que $a \leq M$ para todo $a \in A$.
    Nesse caso dizemos que $M$ é uma \textit{cota superior} de $A$.

    Analogamente se define um conjunto \textit{limitado inferiormente} e \textit{cota inferior}.
\end{mydef}


\begin{mydef}
    Dado um corpo ordenado $(K, +, \cdot)$, um subconjunto $A \subset K$, dizemos que $a \in A$ é o \textit{supremo}  de $A$ se
    \begin{enumerate}
        \item $a$ é uma cota superior de $A$.
        \item Se $b$ é uma cota superior de $A$, então $a \leq b$.
    \end{enumerate}
    Ou seja, $a$ é o menor dos limitantes superiores de $A$ e denotamos $a = \sup A$.

    Analogamente, definimos o \textit{ínfimo} de $A$, denotado por $\inf A$.
\end{mydef}

\begin{mydef}
    Dado um corpo ordenado $(K, +, \cdot)$, dizemos que tal corpo é \textit{completo} se, e somente se, todo subconjunto não vazio $A \subset K$ que é limitado superiormente possui um supremo.
\end{mydef}

\begin{mydef}
    Tomamos por axioma a existência de um corpo ordenado completo $(\mathbb{R}, +, \cdot)$, chamado de \textit{corpo dos números reais}.
\end{mydef}

\begin{mytheo}
    \textbf{Desigualdade de Bernoulli:} Para todo $x \in \mathbb{R}, x \geq -1$ e todo $n \in \mathbb{N}$, temos que $(1+x)^n \geq 1 + nx$.
\end{mytheo}

\begin{mytheo}
    \textbf{Desigualdade triangular:}Se $x,y \in \mathbb{R}$ então $|x+y| \leq |x| + |y|$.
\end{mytheo}

\begin{mytheo}
    \textbf{Teorema dos subintervalor encaixados:} Dada uma sequência de intervalos limitados e fechados $I_1 \subset I_2 \subset \ldots \subset I_n \subset \ldots$,
    $I_n = [a_n, b_n]$ então existe um número real $c$ tal que $c \in I_n$ para todo $n \in \mathbb{N}$.
\end{mytheo}

\begin{mytheo}
    O conjunto dos números reais não é enumerável.
\end{mytheo}

\subsection{Sequências} \label{sec:seq}

Aqui será apresentada a definição de sequência de números reais e a noção de limite em sua forma mais simples, sendo o limite de uma sequência.
Assim, definiremos os conceitos de sequências convergentes, divergentes e sequências de Cauchy. Em seguida, apresentaremos alguns teoremas relacionados com tais conceitos.


\vskip 0.5cm


\begin{mydef}
    Uma sequência de números reais é uma função $x: \mathbb{N} \to \mathbb{R}$ que associa a cada $n \in \mathbb{N}$ um número real $x_n$.
    Denota-se a sequência por $(x_n)$ ou $(x_n)_{n \in \mathbb{N}}$.
\end{mydef}

\begin{mydef}
    Uma sequência $(x_n)$ é dita \textit{limitada superiormente} se existe $M \in \mathbb{R}$ tal que $x_n \leq M$ para todo $n \in \mathbb{N}$.

    Analogamente, definimos uma sequência \textit{limitada inferiormente}.
\end{mydef}

\begin{mydef}
    Dizemos que o número real $a$ é limite da sequência $(x_n)$ se, e somente se, para todo $\epsilon > 0$ existe $N \in \mathbb{N}$ tal que
    $|x_n - a| < \epsilon$ para todo $n \geq N$.

    Ou seja,
    \begin{equation}
        \lim_{n \to \infty} x_n = a \iff \forall \epsilon > 0, \exists N \in \mathbb{N} \text{ tal que } n \geq N \implies |x_n - a| < \epsilon.
    \end{equation}
\end{mydef}

\begin{mydef}
    Uma sequência $(x_n)$ é dita \textit{convergente} se existe um número real $a$ tal que $\lim_{n \to \infty} x_n = a$, caso contrário, dizemos que a sequência é \textit{divergente}.
\end{mydef}

\begin{mytheo}
    \textbf{Unicidade do limite:} Se a sequência $(x_n)$ é convergente, então o limite é único.
\end{mytheo}

\begin{mydef}
    Dada uma sequência $(x_n)$, definimos como subsequência a restrinção de $(x_n)$ a um subconjunto infinito de $\mathbb{N}' \subset \mathbb{N}$ onde
    $\mathbb{N}' = \{ n_1 < n_2 < \ldots < n_k < \ldots \}$. Escrevemos $(x_{n})_{n \in \mathbb{N}'}$.
\end{mydef}

\begin{mytheo}
    Seja $(x_n)$ uma sequência onde $\lim x_n = a$, etão toda subsequência de $(x_n)$ também converge para $a$.
\end{mytheo}

\begin{mytheo}
    Toda sequência convergente é limitada.
\end{mytheo}

\begin{mydef}
    Uma sequência $(x_n)$ é dita \textit{monótona} se $x_{n+1} \geq x_n$ para todo $n \in \mathbb{N}$.
\end{mydef}


\begin{mytheo}
    Toda sequência monótona e limitada é convergente.
\end{mytheo}

\begin{mytheo}
    \textbf{Teorema de Bolzano-Weierstrass:} Toda sequência limitada possui uma subsequência convergente.
\end{mytheo}

\begin{mytheo}
    Seja $(x_n)$ uma sequência monótona que possui uma subsequência $(x_k)_{k \in \mathbb{N}'}$ convergente. Então, a sequência $(x_n)$ também é convergente.
\end{mytheo}


\begin{mytheo}
    \textbf{Teorema do sanduíche:} Sejam $(x_n), (y_n), (z_n)$ sequências de números reais tais que $x_n \leq y_n \leq z_n$ para $n \in \mathbb{N}$ suficientemente grande.
    Se $\lim x_n = \lim z_n = a$, então $\lim y_n = a$.
\end{mytheo}


\begin{mydef}
    Uma sequência $(x_n)$ é dita \textit{de Cauchy} se, para todo $\epsilon > 0$ existe $N \in \mathbb{N}$ tal que $|x_n - x_m| < \epsilon$ para todo $n, m \geq N$.
\end{mydef}


\begin{mytheo}
    Toda sequência convergente é de Cauchy.

    \begin{corollary}
        Toda sequência de Cauchy é limitada.
    \end{corollary}

\end{mytheo}


\begin{mytheo}
    Toda sequência de Cauchy de números reais é convergente.
\end{mytheo}

\subsection{Topologia na reta}

Em seguida, apresentaremos conceitos básicos relacionados à topologia dos números reais. A topologia é uma área da matemática que estuda as propriedades de conjuntos.
Dessa forma, definiremos conceitos como conjuntos abertos, fechados, ponto interior, ponto aderente, ponto de acumulação, vizinhança e conjuntos compactos.


\begin{mydef}
    Dado um conjunto $X \subset \mathbb{R}$, um ponto $a \in X$ é dito ponto interior de $X$ quando existe $\epsilon >0$ tal que
    o intervalo aberto $(a-\epsilon, a+\epsilon) \subset X$.

    O conjunto dos pontos interiores de $X$ chama-se \textit{interior} de $X$ e é denotado por $\text{int}(X)$.
\end{mydef}

\begin{mydef}
    Um conjunto $X \subset \mathbb{R}$ é dito \textit{aberto} se todo ponto de $X$ é um ponto interior de $X$. Ou seja, $X = \text{int}(X)$.
\end{mydef}

\begin{mydef}
    Dado $X \subset \mathbb{R}$ e $a \in X$, dizemos que $a$ é ponto aderente de $X$ se, e somente se, $\exists (x_n) \subset X$ tal que $\lim x_n = a$.

    Note que todo ponto $b$ de $X$ é um ponto aderente de $X$, pois podemos tomar a sequência constante $x_n = b$ para todo $n \in \mathbb{N}$.

    Chama-se \textit{fecho} de $X$ o conjunto dos pontos aderentes de $X$ e é denotado por $\overline{X}$ e tem-se $X \subset \overline{X}$.
\end{mydef}

\begin{mydef}
    Um conjunto $X \subset \mathbb{R}$ é dito \textit{fechado} se $\overline{X} = X$. Ou seja, $X$ contém todos os seus pontos aderentes.
\end{mydef}

\begin{mydef}
    Dado $\epsilon \in \mathbb{R}, \epsilon >0$, definimos o \textit{bola aberta} de raio $\epsilon$ centrada em $a$ como o conjunto $B(a, \epsilon) = \{ x \in \mathbb{R}; |x-a| < \epsilon \}$.

    Dizemos que $B(a, \epsilon)$ é uma vizinhança de $a$.
\end{mydef}

\begin{mytheo}
    Um ponto $a$ é ponto aderente de $X$ se, e somente se, toda vizinhança de $a$ contém um ponto de $X$.
\end{mytheo}


\begin{mytheo}
    Um conjunto $F \subset \mathbb{R}$ é fechado se, e somente se, seu complementar $A = \mathbb{R} \backslash F$ é aberto.
\end{mytheo}

\begin{mydef}
    Um ponto $a \in \mathbb{R}$ é \textit{ponto de acumulação} de um conjunto $X \subset \mathbb{R}$ se, e somente se, toda vizinhança
    de $a$ contém algum ponto de $X$ diferente de $a$.

    Se $a$ não é ponto de acumulação de $X$, então $a$ é um \textit{ponto isolado} de $X$.
\end{mydef}

\begin{mytheo}
    Dados $X \subset \mathbb{R}$ e $a \in \mathbb{R}$, as seguintes afirmações são equivalentes:
    \begin{enumerate}
        \item $a$ é ponto de acumulação de $X$.
        \item Existe uma sequência $(x_n) \subset X\backslash \{a\}$ tal que $\lim x_n = a$.
        \item Toda vizinhança de $a$ contém infinitos pontos de $X$.
    \end{enumerate}
\end{mytheo}


\begin{mydef}
    Um conjunto $X \subset \mathbb{R}$ é dito \textit{compacto} quando é limitado e fechado.
\end{mydef}

\begin{mytheo}
    Um conjunto $X \subset \mathbb{R}$ é compacto se, e somente se, toda sequência $(x_n) \subset X$ possui uma subsequência convergente para um ponto de $X$.
\end{mytheo}

\subsection{Limites de funções}

Finalmente, expandiremos o conceito de limite apresentado na seção \ref{sec:seq} para funções reais de variável real.

\vskip 0.5cm


\begin{mydef}
    Seja $X \subset \mathbb{R}$, $f: X \to \mathbb{R}$ uma função e $a \in X'$ ponto de acumulação de $X$. Dizemos que $L \in \mathbb{R}$ é o \textit{limite de $f(x)$
        quando $x$ tende à $a$} se, e somente se, para todo $\epsilon > 0$ existe $\delta > 0$ tal que $x \in X$ e $0 < |x-a| < \delta \Rightarrow |f(x) - L| < \epsilon$.

    Simbolicamente, temos
    \begin{equation}
        \lim_{x \to a} f(x) = L \iff \forall \epsilon > 0, \exists \delta > 0; x \in X,\quad  0 < |x-a| < \delta \implies |f(x) - L| < \epsilon.
    \end{equation}
\end{mydef}

\begin{mytheo}
    Sejam $f,g: X \to \mathbb{R}, a \in X', \lim_{x \to a} f(x) =L$ e $\lim_{x \to a} g(x) = M$. Se $L < M$ então existe $\delta > 0$ tal que
    $x \in X, 0 < |x-a| < \delta \implies f(x) < g(x)$.
\end{mytheo}

\begin{mytheo}
    \textbf{Teorema do sanduíche para funções:} Sejam $f,g,h: X \to \mathbb{R}, a \in X'$ e $\lim_{x \to a} f(x) = \lim_{x \to a} h(x) = L$.
    Se $f(x) \leq g(x) \leq h(x)$ para $x \in X\backslash\{a\}$, então $\lim_{x \to a} g(x) = L$.
\end{mytheo}

\begin{mytheo}
    Sejam $f:X \to \mathbb{R}, a \in X'$. Para que $\lim_{x \to a}f(x) = L$ é necessário e suficiente que, para toda sequência $(x_n) \subset X\backslash\{a\}$ tal que
    $\lim x_n = a$, tem-se $\lim f(x_n) = L$.
\end{mytheo}

\begin{mytheo}
    \textbf{Unicidade do limite:} Sejam $f: X \to \mathbb{R}, a \in X'$. Se $\lim_{x \to a} f(x) = L$ e $\lim_{x \to a} f(x) = M$, então $L = M$.
\end{mytheo}

\begin{mytheo}
    Sejam $f: X \to \mathbb{R}, a \in X'$. Se existe $\lim_{x \to a} f(x)$ então $f$ é limitada em uma vizinhança de $a$.
    Ou seja, existem $M > 0$ e $\delta > 0$ tal que $x \in X, 0 < |x-a| < \delta \implies |f(x)| \leq M$.
\end{mytheo}

\end{document}
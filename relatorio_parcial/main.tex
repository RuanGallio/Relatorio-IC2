% make an article with a4 paper and 12pt font size
\documentclass[a4paper,12pt]{article}
\usepackage{adjustbox}
\usepackage{amsthm}
\usepackage{amsmath} % Import the amsmath package for \text
\usepackage{amssymb} % Import the amssymb package for \mathbb
\usepackage{etoolbox}

\newcounter{definition}[subsection]
\newenvironment{mydef}{%
  \refstepcounter{definition}%
  \textbf{Definição \thesubsection.\thedefinition. }%
  \ignorespaces
}{%
  \par\addvspace{\topsep}
}

\newcounter{theorem}[subsection]
\newenvironment{mytheo}{%
  \refstepcounter{theorem}%
  \textbf{Teorema \thesubsection.\thetheorem. }%
  \ignorespaces
}{%
  \par\addvspace{\topsep}
}

\newcounter{corollary}[subsection]
\newenvironment{corollary}{%
  \refstepcounter{corollary}%
  \textbf{Corolário \thecorollary. }%
  \ignorespaces
}{%
  \par\addvspace{\topsep}
}


% make a demonstration environment. Add black square at the end of the demonstration
\newenvironment{demonstration}{%
  \par\addvspace{\topsep}
  \textbf{Demonstração. }%
  \ignorespaces
}{%
  \par\addvspace{\topsep}
  \hfill $\blacksquare$
}

% This code ensures that theorem and definition counters are reset with every new section
\makeatletter
\@addtoreset{theorem}{subsection}
\@addtoreset{definition}{subsection}
\@addtoreset{corollary}{theorem}
\makeatother


\begin{document}

% align right
\begin{flushleft}
    \textbf{\large PRÓ-REITORIA DE PESQUISA E PÓS-GRADUAÇÃO}
    \linebreak
    \linebreak
    \textbf{\large COORDENAÇÃO LOCAL DO PROGRAMA DE INICIAÇÃO CIENTÍFICA}
\end{flushleft}

\begin{center}
    \textbf{RELATÓRIO}

    \textbf{ANEXO I}
\end{center}

\section{Resumo}

Neste trabalho pretendemos apresentar um resultado de existência e unicidade de
solução para um problema envolvendo a equação da onda. Será estudado classes de
problemas com dados iniciais regulares. Precisamente, consideraremos os dados
iniciais em espaços de funções m vezes continuamente diferenciáveis. Para isso será
efetuado levantamento bibliográfico preliminar sobre tópicos de Análise Real e
Espaços Métricos. Serão estudados aspectos relativos ao espaço das funções
contínuas com métricas apropriadas. O resultado principal será provado via Séries de
Fourier.

\section{Objetivos}

\begin{itemize}
    \item Complementar a formação do discente, visando um perfil científico.

    \item Estudar aspectos relativos aos espaços das funções contínuas, e das funções
          continuamente diferenciáveis, sob o ponto de vista de espaços métricos.

    \item Provar um teorema que estabeleça condições para a existência de solução
          para a equação de propagação de ondas acústicas.

    \item Propiciar que o aluno aprenda aspectos teóricos sobre Análise Real, Espaços
          Métricos e aplicá-los no estudo da equação da onda.

    \item Apresentar os resultados finais no Encontro Anual de Iniciação Científica,
          Tecnológica e Inovação.
\end{itemize}

\section{Cronograma de atividades}

% Make a table from september 2023 to August 2024 as colluwns and 5 rows.
% Add each month as title of the column and the activities as rows.

\begin{adjustbox}{width={\textwidth},totalheight={\textheight},keepaspectratio}%
    \centering
    \begin{tabular}{|c|c|c|c|c|c|c|c|c|c|c|c|c|}
        \hline
        \textbf{Atividades}          & \textbf{Set} & \textbf{Out} & \textbf{Nov} & \textbf{Dez} & \textbf{Jan} & \textbf{Fev} & \textbf{Mar} & \textbf{Abr} & \textbf{Mai} & \textbf{Jun} & \textbf{Jul} & \textbf{Ago} \\
        \hline
        Estuda das referências       & X            & X            & X            & X            & X            & X            & X            & X            &              &              &              &              \\
        \hline
        Escruta do relatório parcial &              &              &              &              & X            & X            &              &              &              &              &              &              \\
        \hline
        Formulação do resultado      &              &              &              &              &              &              & X            & X            & X            & X            &              &              \\
        \hline
        Prova do resultado           &              &              &              &              &              &              &              & X            & X            & X            & X            &              \\
        \hline
        Escrita do relatório final   &              &              &              &              &              &              &              &              &              & X            & X            & X            \\
        \hline
    \end{tabular}
\end{adjustbox}


\section{Materiais e métodos}

\section{Descrição dos principais resultados}

\subsection{Teorema de existência de solução da equação da onda}
Nesta subseção, apresentaremos definições bases sobre Equações Diferenciais Parciais (EDP), em específico, a equação da onda e o fundamento teórico
necessário para a prova do teorema de existência e unicidade de solução para o problema de valor inicial e de fronteira (PVIF) associado à equação da onda.


\begin{mydef}
    Uma função $f: \mathbb{R} \to \mathbb{R}$ é \textit{periódica} de período $T$ se, e somente se, $f(x + T) = f(x)$ para todo $x \in \mathbb{R}$.
\end{mydef}

\begin{mydef}
    Dada uma série de funções $\sum_{n=1}^{\infty} u_{n}(x)$ onde $u_{n}(x): I \to \mathbb{R}$ \textit{converge pontualmente} se, para cada $x_0 \in I$ fixado,
    a série numérica $\sum_{n=1}^{\infty} u_{n}(x_0)$ é convergente. Ou seja, dados $\epsilon > 0$ e $x_0 \in I$, existe $N(\epsilon, x_0) \in \mathbb{N}$ tal que
    \begin{equation}
        \left| \sum_{j=n}^{m} u_{j}(x_0) \right| < \epsilon
    \end{equation}
    para todo $m \geq n \geq N(\epsilon, x_0)$.
\end{mydef}

\begin{mydef}
    Dada uma série de funções $\sum_{n=1}^{\infty} u_{n}(x)$ onde $u_{n}(x): I \to \mathbb{R}$ \textit{converge uniformemente} se, para cada $\epsilon > 0$, existe
    $N(\epsilon) \in \mathbb{N}$ (ou seja, independente de $x \in I$) tal que
    \begin{equation}
        \left| \sum_{j=n}^{m} u_{j}(x) \right| < \epsilon
    \end{equation}
    para todo $m \geq n \geq N(\epsilon)$ e para todo $x \in I$.
\end{mydef}

\begin{mytheo}
    \textbf{Teste M de Weierstrass:} Seja $\sum_{n=1}^{\infty}u_{n}(x)$ uma série de funções $u_{n}: I \to \mathbb{R}$ definidas em um subconjunto $I$ de $\mathbb{R}$. Suponha que existam constantes $M_{n}$ tais que
    \begin{equation}
        |u_{n}(x)| \leq M_{n}|, \qquad \text{para todo  }x \in I,
    \end{equation}

    e que a série numérica $\sum_{n=1}^{\infty}M_{n}$ convirja. Então a série de funções  $\sum_{n=1}^{\infty}u_{n}(x)$ tem convergência uniforme em $I$.
\end{mytheo}

\begin{mydef}
    Dada uma função $f: \mathbb{R} \to \mathbb{R}$, periódica, de período $2L$, integrável e absolutamente integrável, podemos expressá-la como uma \textit{série de Fourier}
    da seguinte forma
    \begin{equation}
        f(x) = \frac{1}{2}a_0 + \sum_{n=1}^{\infty} a_{n}\cos\left(\frac{n\pi x}{L}\right) + b_{n}\sin\left(\frac{n\pi x}{L}\right).
    \end{equation}
\end{mydef}

\begin{mydef}
    Dada uma função $f$ que possa se expressa como um série de Fourier, os coeficientes $a_{n}$ e $b_{n}$ são chamados de \textit{coeficientes de Fourier} de $f$ e são dados por
    \begin{align}
        a_{n} & = \frac{1}{L}\int_{-L}^{L}f(x)\cos\left(\frac{n\pi x}{L}\right)\, dx; \quad n \geq 0 \\
        b_{n} & = \frac{1}{L}\int_{-L}^{L}f(x)\sin\left(\frac{n\pi x}{L}\right)\, dx; \quad n \geq 1
    \end{align}
\end{mydef}


\begin{mydef}
    A equação da onda é dada pelo Problema de Valor Inicial e de Fronteira (PVIF)

    \begin{equation}
        \begin{cases}
            u_{tt} = c^{2}u_{xx}, & \text{em } \mathbb{R}      \\
            u(0,t) = u(L,t) = 0,  &                            \\
            u(x, 0) = f(x),       & \text{em } 0 \leq x \leq L \\
            u_t(x, 0) = g(x),     & \text{em } 0 \leq x \leq L
        \end{cases}
    \end{equation}
\end{mydef}

\begin{mydef}
    This is the first mydefinition.
\end{mydef}

\begin{mydef}
    This is the first mydefinition.
\end{mydef}


\subsection{Conjuntos e funções}

\begin{mydef}
    This is the first mydefinition.
\end{mydef}


\subsection{Cardinalidade de conjuntos}
\begin{mydef}
    Dada um conjunto $\mathbb{N}$ e uma função $s:\mathbb{N} \to \mathbb{N}$, chamamos os elementos
    desse conjunto de números naturais se, e somente se, $s$ satisfaz os seguintes axiomas \textit{(chamados Axiomas de Peano)}
    \begin{enumerate}
        \item $s: \mathbb{N} \to \mathbb{N}$ é uma função injetora.
        \item $\mathbb{N}\backslash s(\mathbb{N})$ consta de um só elemento; tal elemento é chamado de "um", com símbolo $1$.
              Ou seja, 1 não é sucessor de nenhum número natural.
        \item Se $X \subset \mathbb{N}$ é um subconjunto tal que $1 \in \mathbb{N}$ e,
              para todo $n \in \mathbb{N}$, se $n \in X$, então $s(n) \in X$, então $X = \mathbb{N}$.
    \end{enumerate}
\end{mydef}

\begin{mydef}
    Em $\mathbb{N}$, definimos a adição de dois números naturais como a operação $+$ sobre um par $(m,n)$ da seguinte forma:
    \begin{enumerate}
        \item $m + 1 = s(m)$
        \item $m + s(n) = s(m + n)$, ou seja, $m + (n+1) = (m+n)+1$
    \end{enumerate}
\end{mydef}


\begin{mydef}
    Em $\mathbb{N}$, definimos a multiplicação de dois números naturais como a operação $\cdot$ sobre um par $(m,n)$ da seguinte forma:
    \begin{enumerate}
        \item $m \cdot 1 = m$
        \item $m \cdot (n+1) = m\cdot n + m$
    \end{enumerate}
\end{mydef}

\begin{mytheo}
    Sobre as operações de adição e multiplicação de números naturais, temos as seguintes propriedades:
    \begin{itemize}
        \item \textit{associatividade}: $(m + n) + p = m + (n + p)$ e $(m \cdot n) \cdot p = m \cdot (n \cdot p)$
        \item \textit{comutatividade}: $m + n = n + m$ e $m \cdot n = n \cdot m$
        \item \textit{distributividade}: $m \cdot (n + p) = m \cdot n + m \cdot p$
        \item \textit{lei do corte}: $m + p = n + p \Rightarrow m = n$ e $m \cdot p = n \cdot p \Rightarrow m = n$
    \end{itemize}
\end{mytheo}

\begin{mytheo}
    \textbf{Princípio da boa-ordenação} Todo subconjunto não vazio $A \subset \mathbb{N}$ possui um menor elemento.


    \begin{demonstration}
        \textbf{DEMONSTRAR}
    \end{demonstration}
\end{mytheo}

\begin{mydef}
    Definimos como o conjunto de inteiros menores ou iguais a $n$ como $I_n = \{ p \in \mathbb{N}; p \leq n \}$.
\end{mydef}

\begin{mydef}
    Um conjunto $X$ é dito finito se é vazio ou, então, se existe uma função bijetora $f: I_n \to X$.
    A bijeção $f$ chama-se \textit{contagem} de $X$ e o número $n$ chama-se \textit{número de elementos} ou \textit{número cardinal} de $X$.
\end{mydef}

\begin{mytheo}
    Seja $X$ um conjunto finito e $f:X \to X$ uma função. Então, $f$ é injetora se, e somente se, $f$ é sobrejetora.
    \begin{demonstration}
        \textbf{DEMONSTRAR}
    \end{demonstration}
\end{mytheo}

\begin{mydef}
    Definir conjunto limitado
\end{mydef}

\begin{mytheo}
    Um subconjunto $X$ de $\mathbb{N}$ é finito se, e somente se, $X$ é limitado.
\end{mytheo}

\begin{mydef}
    Um conjunto $X$ é dito infinito se não é finito.
    Ou seja, $X$ é infinito se, e somente se, não existe uma função bijetora $f: I_n \to X$ para nenhum $n \in \mathbb{N}$.
\end{mydef}

\begin{mytheo}
    Se $X$ é um conjunto infinito, então existe uma função injetora $f: \mathbb{N} \to X$.

    \begin{corollary}
        Um conjunto $X$ é infinito se, e somente se, existe uma bijeção $\psi: X \to Y$ sobre um subconjunto próprio $Y \subset X$.
    \end{corollary}
\end{mytheo}

\begin{mydef}
    Um conjunto $X$ é dito \textit{enumerável} quando é finito ou quando existe uma bijeção $f: \mathbb{N} \to X$.
    Neste caso, $f$ chama-se \textit{enumeração} de $X$.
\end{mydef}

\begin{mytheo}
    Todo subconjunto $X \subset \mathbb{N}$ é enumerável.

    \begin{demonstration}
        \textbf{DEMONSTRAR}
    \end{demonstration}


    \begin{corollary}
        O produto cartesiano de dois conjuntos enumeráveis é enumerável.
    \end{corollary}

\end{mytheo}


\subsection{Números reais}

\subsection{Sequências e séries}
\subsection{Topologia na reta}
\subsection{Limites de funções}



\end{document}